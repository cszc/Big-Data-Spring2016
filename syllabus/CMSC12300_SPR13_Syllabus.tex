\documentclass[11pt]{article}
\usepackage{fullpage}
\usepackage{titlesec}
\usepackage{titletoc}
\usepackage{fancybox}
\usepackage{multirow}
\usepackage[usenames,dvipsnames]{color}
\usepackage{colortbl}
\usepackage{rotating}
\usepackage{verbatim}
\usepackage{lscape}
\usepackage{longtable}
\usepackage{url}
\usepackage{todonotes}


\newcommand{\emptycell}{\cellcolor[gray]{0.9}}
\newcommand{\TODO}[1]{\todo[inline,color=yellow!40]{TODO: #1}}

%opening
\title{CMSC 12300\\Computer Science with Applications 3\\{\large\color{red}Tentative Syllabus}\\{\large\emph{Spring 2013 Quarter}}}
\author{Department of Computer Science\\University of Chicago}
\date{Last updated: 4/17/2013}

\setcounter{tocdepth}{1}

\titlecontents{section}
[1em]
{\sffamily}
{}
{}
{\titlerule*[0.5pc]{.}\contentspage\hspace*{1em}}
\renewcommand\contentsname{Contents of this Document}
\begin{document}

\maketitle

\thispagestyle{empty}

\begin{center}
\vspace{-2ex}
\begin{minipage}{0.6\textwidth}
\begin{center}
Lecture: TuTh 4:00-5:20 (Ryerson 276)

Lab: W 3:30-4:50 (JRL A01C)
\vspace{2ex}

\textbf{Instructor}


Borja Sotomayor

\url{borja@cs.uchicago.edu}

Ryerson 151

\vspace{1em}
\textbf{TA}
\vspace{1ex}

Gustav Larsson

\url{larsson@cs.uchicago.edu}

Ryerson 177

\end{center}

\end{minipage}

\end{center}

\vspace{2ex}

\titleformat{\section}[block]
{\filcenter\normalfont\sffamily}
{}{0em}{}

\begin{center}
\shadowbox{
\begin{minipage}{0.6\textwidth}
\tableofcontents
\end{minipage}
}

\vspace{2ex}

\textbf{Website:}\\
\url{http://www.classes.cs.uchicago.edu/archive/2013/spring/12300-1/}
\end{center}

\titleformat{\section}[block]
{\large\sffamily}
{}{0em}{\titlerule\\\bfseries}

\titleformat{\subsection}[block]
{\normalfont\sffamily\bfseries}
{}{0em}{}

\pagebreak

\section{Course description}

This course is the third in a three-quarter sequence that teaches 
computational thinking and skills to students in the sciences, 
mathematics, economics, etc. The course revolves around core ideas 
behind the management and computation of large volumes of data ("Big Data"). 
Topics include (1) Statistical methods for large data analysis, 
(2) Parallelism and concurrency, including models of parallelism and 
synchronization primitives, and (3) Distributed computing, including
distributed architectures and the algorithms and techniques that 
enable these architectures to be fault-tolerant, reliable, and scalable.

Students will continue to use R, and will also learn C++ and distributed 
computing tools and platforms, including Amazon AWS and Hadoop. This 
course includes a project where students will have to formulate 
hypotheses about a large dataset, develop statistical models to test 
those hypothesis, implement a prototype that performs an initial
exploration of the data, and a final system to process the entire dataset.

CMSC 12200, or instructor's consent, is a prerequisite for taking this course.

\section{Course organization}

The development of a project requiring the analysis of a large dataset
accounts for the majority of the grade in this course. Students will 
be provided with a list of possible projects and datasets, but are
strongly encouraged to pursue a project that is of interest to them.
Projects can be developed individually or in pairs. To successfully 
complete their project, students must understand 
fundamental concepts in distributed computing, parallelism,
concurrency, and statistical methods suited for large volumes
of data. 

The course meets two times a week for lectures that provide this 
conceptual scaffolding. These lectures are complemented by labs
and programming assignments
that allow students to put these concepts into practice through
the use of distributed computing tools and systems. There are no exams. 
The course calendar, including the 
contents of each lecture and project deadlines, is shown in Table~\ref{tab:calendar}.


\subsection{Project}

Throughout the quarter, students will develop a programming project
that requires working with a large dataset. For the purposes of this
class, we define a ``large dataset'' as one that (1) will not
fit in the hard drive of an average desktop/laptop computer, or (2)
will be processed in such a way that a single computer would take
days or weeks to produce any results. The size of these datasets will
typically be in the hundreds of gigabytes.

We expect students will approach the dataset in one of two ways: (1) you
will formulate a series of hypotheses on the data and will write a
project that analyzes the entire dataset to prove those hypotheses, or
(2) you will develop an application that requires access to the
entire dataset to work (e.g., an application that can interactively
answer non-trivial queries or generate interesting visualizations
on the dataset).

Projects can be developed individually or in pairs, although we
strongly encourage you to work in pairs. Projects will be developed 
on GitHub, and all submissions will be done through your
GitHub repository.

The project will be broadly divided into two phases: the data exploration
and prototyping phase, and the final submission phase. In the first
phase, students are expected to take a subset of the dataset and do
some preliminary data exploration using the techniques and tools
learned in CMSC 12200. In the second phase, students will take
what they learned from the data exploration to design and implement
a final version that operates on the entire dataset. This second phase
will involve using some of the tools and techniques covered in this
course specifically.

The project will be divided into the following specific milestones.
Specific guidelines and expectations for each of these milestones
will be provided before each deadline.

\begin{itemize}
 \item \textbf{Project proposal} (due Monday, April 15). A one-page proposal
describing:

\begin{enumerate}
\item The dataset you will be using.
\item A description of how you intend to perform the data exploration 
(including what tools and techniques you expect to use)
and the ideal outcome of your first prototype.
\item A description of the final results you expect to obtain (either
the hypotheses you intend to prove or disprove, or the application
you expect to develop).
\end{enumerate}

 \item \textbf{Project proposal} (in class, Thursday, April 18). You
will give a 10-minute presentation on your project proposal in class.
This will be an informal presentation which will not be graded; the
purpose of the presentation is for everyone in the class to be aware
of all the projects, and to get feedback from your peers (and possibly
identify collaborations between different projects).

 \item \textbf{Prototype} (due Tuesday, April 30). You must push
a prototype of your project to your GitHub repository. Your prototype
must include specific instructions on how to obtain the data necessary
to run the prototype, and on how to run the prototype itself. You
should also include a brief writeup (1-2 pages) describing what
you have learned about the data in the process of exploring the data
and developing a prototype.

 \item \textbf{Prototype presentation} (in class Tuesday, May 7).
You will give a 10-minute demonstration of your prototype in class.
This will be an informal presentation which will not be graded. 

 \item \textbf{Status report} (due Monday, May 20). A brief 1-2 page 
report specifying what progress you've made since presenting your prototype. 
You must also arrange to meet with the instructor during eighth week
to discuss your progress.

 \item \textbf{Final report} (due Tuesday, June 4). You must push
the final version of your project into your repository by this date,
accompanied by an 8-10 page report summarizing the design and implementation
of your project, and the main conclusions you have extracted from the
dataset you worked with. 

 \item \textbf{Final presentation} (in class Tuesday, June 4).
You will give a 10-minute demonstration of your project in class.
This presentation \emph{will} be graded. If possible, we will try
to allocate more time for the presentations at a different (non-class)
time.

\end{itemize}


\begin{landscape}
\sffamily
\setlength{\extrarowheight}{4pt}
\begin{longtable}{|c|cc|c|p{7cm}|p{6cm}|p{2cm}|c|}
\caption{CMSC 12300 Spring 2013 Schedule}\label{tab:calendar}\\
\hline
\textbf{Week} & \multicolumn{2}{c|}{\textbf{Date}} & \multicolumn{ 2}{c|}{\textbf{Lecture}} & \multicolumn{1}{c|}{\textbf{Lab}} & \textbf{Due} \\ \hline
\multirow{3}{*}{1} 	& Tu & Apr 02 	& 1 		& Course Introduction 		& \emptycell  	& \emptycell \\ \cline{ 2- 8}
			& W & Apr 03	& \emptycell 	& \emptycell			& Git \& GitHub	& \emptycell \\ \cline{ 2- 8}
			& Th & Apr 04 	& 2 		& Statistical Methods      	& \emptycell  	& \emptycell \\ \hline\hline

\multirow{4}{*}{2} 	& Tu & Apr 09 	& 3 		& Statistical Methods      	& \emptycell  	& \emptycell \\ \cline{ 2- 8}
			& W & Apr 10	& \emptycell 	& \emptycell			& AWS	 	& \emptycell \\ \cline{ 2- 8}
			& Th & Apr 11 	& 4 		& Example Application I    	& \emptycell  	& \emptycell \\ \hline\hline

\multirow{3}{*}{3} 	& M & Apr 15 	& \emptycell	& \emptycell 			& \emptycell  	& Project \mbox{Proposal} \\ \cline{ 2- 8}			
			& Tu & Apr 16 	& 5 		& Statistical Methods   	& \emptycell  	& \emptycell \\ \cline{ 2- 8}
			& W & Apr 17	& \emptycell 	& \emptycell			& Data Analysis Methods	 	& \emptycell \\ \cline{ 2- 8}
			& Th & Apr 18 	& 6 		& Proposal Presentations	& \emptycell  	& \emptycell \\ \hline\hline

\multirow{3}{*}{4} 	& Tu & Apr 23 	& 7 		& C++				& \emptycell  	& \emptycell \\ \cline{ 2- 8}
			& W & Apr 24	& \emptycell 	& \emptycell			& C++	 	& \emptycell \\ \cline{ 2- 8}
			& Th & Apr 25 	& 8 		& C++			 	& \emptycell  	& \emptycell \\ \hline\hline

\multirow{3}{*}{5} 	& Tu & Apr 30 	& 9 		& Distributed Systems		& \emptycell  	& \emptycell \\ \cline{ 2- 8}
			& W & May 01	& \emptycell 	& \emptycell			& TBD	 	& \emptycell \\ \cline{ 2- 8}
			& Th & May 02 	& 10 		& Distributed Systems	 	& \emptycell  	& \emptycell \\ \hline\hline

\multirow{3}{*}{6} 	& Tu & May 07 	& 11 		& Prototype presentations	& \emptycell  	& Project \mbox{Prototype}\\ \cline{ 2- 8}
			& W & May 08	& \emptycell 	& \emptycell			& TBD	 	& \emptycell \\ \cline{ 2- 8}
			& Th & May 09 	& \emptycell	& Example Application II  	& \emptycell    & \emptycell \\ \hline\hline

\multirow{3}{*}{7} 	& Tu & May 14 	& 12 		& Distributed Systems		& \emptycell  	& \emptycell \\ \cline{ 2- 8}
			& W & May 15	& \emptycell 	& \emptycell			& TBD	 	& \emptycell \\ \cline{ 2- 8}
			& Th & May 16 	& 13 		& Parallelism, Concurrency 	& \emptycell  	& \emptycell \\ \hline\hline

\multirow{3}{*}{8}	& M & May 20 	& \emptycell	& \emptycell    		& \emptycell  	& Status report \\ \cline{ 2- 8}
			& Tu & May 21 	& 14 		& Parallelism, Concurrency	& \emptycell  	& \emptycell \\ \cline{ 2- 8}
			& W & May 22	& \emptycell 	& \emptycell			& TBD	 	& \emptycell \\ \cline{ 2- 8}
			& Th & May 23 	& 15 		& Parallelism, Concurrency 	& \emptycell  	& \emptycell \\ \hline\hline

\multirow{3}{*}{9} 	& Tu & May 28 	& 16 		& Parallel programming with C++11		& \emptycell &  \emptycell  	 \\ \cline{ 2- 8}
			& W & May 29	& \emptycell 	& \emptycell			& TBD	 	& \emptycell \\ \cline{ 2- 8}
			& Th & May 30 	& 17 		& Parallel programming with C++11	 	& \emptycell  	& \emptycell \\ \hline\hline

\multirow{3}{*}{10} 	& Tu & Jun 04 	& 18 		& Project Presentations		& \emptycell  	& Final report \\ \cline{ 2- 8}
			& W & Jun 05	& \emptycell 	& \emptycell			& TBD	 	& \emptycell \\ \cline{ 2- 8}

\end{longtable}

\end{landscape}



\subsection{Labs}

Weekly labs are intended to give you a chance to practice the material
we have covered in class \emph{and} to expose you to resources that
will be useful for your projects. The code you write during the labs will not
be graded.

\subsection{Programming Assignments}

There will be three programming assignments throughout the quarter to reinforce
some of the concepts and technologies covered in class (but which few, if any,
students may end up using in their projects). Although there are
sometimes weeks between programming assignments, we don't expect these
assignments to require more than a few hours each.

Programming assignments will be announced at least one week before they are due.

\section{Books}

This course has no required textbooks.


\section{Grading}

The final grade will be divided as follows:

\begin{itemize}
 \item \emph{Programming Assignments}: 15\% (each assignment weighted equally)
 \item \emph{Project Proposal}: 15\%
 \item \emph{Prototype}: 20\%
 \item \emph{Status Report}: 15\%
 \item \emph{Final Report}: 25\%
 \item \emph{Final Presentation}: 10\%
\end{itemize}


\subsection{Types of grades}

Students may take this course for a quality grade (a ``letter'' grade)
or a pass/fail grade. Students will declare at any time before their final project presentation that,
depending on their final grade, they want to receive a letter grade, a
pass/fail grade or withdraw from the course (a \emph{W} grade). For
example, students can declare ``If my final grade is a C+ or lower, I
will take a \emph{P} (Pass) instead of a letter grade and, if my grade
is an \emph{F}, I wish to take a \emph{W}''. By default, all students
are assumed to be taking the course for a quality grade.

\begin{quote}
Note: \emph{Students taking this course to meet general education requirements must take the course for a letter grade}. 
\end{quote}

\subsection{Late submissions}

All student have two 24-hour extensions that may be applied to
the Project Proposal or the Status Report.
These extensions are all-or-nothing: you cannot use a portion of an extension and have the
rest ``carry over'' to another extension. If extraordinary
circumstances (illness, family emergency, etc.) prevent a student from
meeting a deadline, the student must inform the instructor
\emph{before} the deadline.


\section{Policy on academic honesty}

The University of Chicago has a formal policy on academic honesty that you are expected to adhere to:

\begin{center}
\url{http://studentmanual.uchicago.edu/academic/index.shtml#honesty}
\end{center}

In brief, academic dishonesty (handing in someone else's work as your
own, taking existing code and not citing its origin, etc.) will
\emph{not} be tolerated in this course. Depending on the severity of
the offense, you risk getting a hefty point penalty or being dismissed
altogether from the course. All cases will be referred to the Dean of
Students office, which may impose further penalties, including
suspension and expulsion.

Even so, discussing the concepts necessary to complete assignments is
certainly allowed (and encouraged).  Under \emph{no circumstances}
should you show (or email) another student your code or post your
solution to a web-page or social media site.  If you have discussed
parts of your project with anyone other than your project partner, 
then make sure to say so in your submission (e.g., in a README file 
or as a comment at the top of your source code file). If
you consulted other sources, please make sure you cite these sources.

If you have any questions regarding what would or would not be
considered academic dishonesty in this course, please don't hesitate
to ask the instructor.

\section{Asking questions}
\label{asking}

The preferred form of support for this course is though \emph{Piazza}
(\url{http://www.piazza.com/}), an on-line discussion service that
can be used to ask questions and share useful information with your
classmates. Students will be enrolled in Piazza at the start of the
quarter.

All questions regarding assignments or material covered in class must
be sent to Piazza, and not directly to the instructors or TAs, as this
allows your classmates to join in the discussion and benefit from the
replies to your question. If you send a message directly to the
instructor or the TAs, you will get a gentle reply asking you to send
your question to Piazza. 

Piazza has a mechanism that allows you to ask a private question,
which will be seen only by the instructors and teaching assistants.
This mechanism should be used \emph{only} for questions that apply
uniquely to you.

Additionally, all course announcements will be made through Piazza.
It is your responsibility to check Piazza often to see if there are
any announcements. Please note that you can configure your Piazza account
to send you e-mail notifications every time there is a new post on
Piazza. Just go to your Account Settings, then to Class Settings, 
click on ``Edit Notifications'' under CMSC 12300. We 
encourage you to select either the ``Real Time'' option (get a notification
as soon as there are new posts) or the ``Smart Digest'' option (get
a summary of all the posts sent over the last 1-6 hours -- you can select
the frequency).

%\section{Office hours}

%TODO

%\section{Link round-up}


\end{document}
